\chapter{Comparison of Methods, Protocols and Algorithms}

\section{Protocols}

\subsection{Kerberos}
\subsection{Secure Shell}
\subsection{Pretty Good Privacy}

\section{Methods}

\subsection{Symmetric Cryptography}
\subsection{Asymmetric Cryptography}
\subsection{Secure Socket Layer}
\subsection{Hashing}

\section{Algorithms}

\subsection{AES vs DES3}
AES is explained in details in section \ref{featuer:aes}
Data Encryption Standard (DES) is an outdated symmetric-key method of data encryption. DES has been superseded by the more secure Advanced Encryption Standard (AES) algorithm. AES and DES almost work the same way with exception that AES is modified to be more secure. DES uses 56 bit key while AES can use 128, 192, or 256 bits key. DES encrypts blocks with size of 64 bits while AES encrypts blocks of size 128 bits. DES was widely used until is was proven inadequate. So to compensate the insecurity of DES people started using 3 layers of DES, so if a different key will be used in each layer we can achieve a key of length 168. However, due to the likelihood of a meet-in-the-middle attack, the effective security it provides is only 112 bits. 3DES encryption is obviously slower than plain DES.
\par So we went for AES-256 because is efficient in both software and hardware implementations, uses 256 bits key over the 112 bits one and encrypts bigger blocks than DES 128 bits and 64 bits respectively. AES is much faster than DES3.

\subsection{Self-Signing Certificates vs Public Key Cryptography}
Public Key Cryptography that is explained in section \ref{featuer:pkc} can be used to achieve authenticity by encrypting the message using the sender's private key, thus the receiver once he decrypts using the sender's key he will know that is sent form the sender. So in order to send data through a secure channel we use used SSL which utilizes public key cryptography to encrypt the data. So if SSL uses public key cryptography to encrypt the data, why is a certificate necessary ?
\par The technical answer to that question is that a certificate is not really necessary, the data is secure and cannot easily be decrypted by a third party. However, certificates do serve a crucial role in the communication process. The certificate, signed by a trusted Certificate Authority (CA), ensures that the certificate holder is really who he claims to be. Without a trusted signed certificate, your data may be encrypted, however, the party you are communicating with may not be whom you think. Without certificates, impersonation attacks would be much more common. As theses certificates contains information about the owner of the certificate, like e-mail address, owner's name, certificate usage, duration of validity, resource location or Distinguished Name (DN) which includes the Common Name (CN) such as web site address or e-mail address, and the certificate ID of the person who certifies (signs) this information. It contains also the public key and finally a hash to ensure that the certificate has not been tampered with.