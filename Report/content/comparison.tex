\chapter{Comparison of Methods, Protocols and Algorithms}

\section{Protocols}

\subsection{Kerberos}
It is a network authentication protocol where it is used for user authentication to provide a secure communications between users over an in-secure network. First, a user would contact a verified Key Distribution Center (KDC) for authentication then the user receives what is called a ticket-granting ticket (TGT) which contains the user's information encrypted. The user then sends the TGT to the required server. The server then contacts the same TDC to verify the received TGT from the user. If the TGT is verified, it means that the user is authenticated.

\subsubsection{Disadvantages:}
\begin{itemize}
    \item It requires available connectivity to the KDC.
    \item The KDC has to be trusted by the user and the server.
\end{itemize}

\subsection{Secure Shell}
Secure Shell (SSH) is a network protocol that allows a computer to remotely-connect to another computer in a secure encrypted connection. The protocol uses public-private key pair for authentication. The user's public key is saved on the computers for all users wishing to connect to it. The protocol then sends an encrypted challenge that is required to be decrypted using the private key of the user. If the user's private key is matching the public key used, the user would be able to decrypt the challenge and sends it back to the server so the server can check the validity of the decrypted challenge.

\subsubsection{Disadvantages:}
\begin{itemize}
    \item The user has to verify and trust the server for the connection to be added for the user's trusted hosts.
    \item A connection has to be established between the user and the server first for verification before the actual data transfer between them.
\end{itemize}

\subsection{Pretty Good Privacy}
Pretty Good Privacy (PGP) is an encryption/decryption protocol for sending encryption keys over a network between parties. It uses symmetric keys for symmetric encryption of data and then encrypts the keys using the public key of the requested user to communicate with. The receiving user would then decrypt the keys using their own private key.

In this project, PGP is adopted for key sharing between the users and the server for code encryption and decryption as it is secure due to using the public-private key encryption. It is also easy to be used. The protocol is tweaked such that it generates a new public-private key for each new user so that it doesn't require any server authentication first. Refer to \ref{sec:pgp_info_1} and \ref{sec:pgp_info_2} for detailed information.

\section{Methods}

\subsection{Symmetric Cryptography vs Asymmetric Cryptography}
Symmetric cryptography means that one key is shared between parties where that key is used for encrypting data and the same key would be used for decrypting the data and it uses one of the symmetric encryption algorithms for that. Examples for that approach is AES and DES3 which both are explained in the next section.

\subsubsection{Advantages:}
\begin{itemize}
    \item It is fast so it can be used in encrypting/decrypting large amount of data.
\end{itemize}

\subsubsection{Disadvantages:}
\begin{itemize}
    \item It is not very secure as it requires sharing the key among parties involved so the key can get compromised.
\end{itemize}

Asymmetric cryptography means that different keys are used for encryption and decryption, that is, one key to be used for encryption and another key would be used for decryption. This method is more secure that symmetric cryptography because it removes the burden of sharing keys between parties as the compromise of one key would not affect the other key. Example for that approach is public-key cryptography which is explained in the next section.

\subsubsection{Advantages:}
\begin{itemize}
    \item It is secure as the compromise of one key would not affect the other key.
\end{itemize}

\subsubsection{Disadvantages:}
\begin{itemize}
    \item It is slow so it can be used for small amount of data.
\end{itemize}

In this project, both methods are used in different parts. Symmetric cryptography is used for encrypting codes in the system. Public-key cryptography is used for encrypting the keys used for symmetric cryptography.

\subsection{Secure Socket Layer vs Hashing}
Secure Socket Layer (SSL) is the process of providing a security layer between a user and a server such that transferred data between both data would be secured against network sniffing or man-in-the-middle attacks. It works by encrypting the data sent between both parties and using the concept of cerificates to verify the server's identity to the user. Public-key cryptography is used for the generation and verification of these certificates.

Hashing is the process of mapping values to other values using a function. One-way hashing is the hashing used in one direction only such that a function would be used on plain value to get hashed value, however, there is no function to go from the hashed value to the plain value.

In this project, both methods are used in the following way. SSL is used to protect the data submitted by the user to the server such that no attacker would be able to get the user's password or passphrase. Hashing is used on the user's password when saved to that database, such that the user's plain password is not to be saved anywhere on the system. Verification is done by hashing the submitted password by the user at login time and comparing the hashed value with the saved value in the database.

\section{Algorithms}

\subsection{AES vs DES3}
AES is explained in details in section \ref{featuer:aes}
Data Encryption Standard (DES) is an outdated symmetric-key method of data encryption. DES has been superseded by the more secure Advanced Encryption Standard (AES) algorithm. AES and DES almost work the same way with exception that AES is modified to be more secure. DES uses 56 bit key while AES can use 128, 192, or 256 bits key. DES encrypts blocks with size of 64 bits while AES encrypts blocks of size 128 bits. DES was widely used until is was proven inadequate. So to compensate the insecurity of DES people started using 3 layers of DES, so if a different key will be used in each layer we can achieve a key of length 168. However, due to the likelihood of a meet-in-the-middle attack, the effective security it provides is only 112 bits. 3DES encryption is obviously slower than plain DES.
\par So we went for AES-256 because is efficient in both software and hardware implementations, uses 256 bits key over the 112 bits one and encrypts bigger blocks than DES 128 bits and 64 bits respectively. AES is much faster than DES3.

\subsection{Self-Signing Certificates vs Public Key Cryptography}
Public Key Cryptography that is explained in section \ref{featuer:pkc} can be used to achieve authenticity by encrypting the message using the sender's private key, thus the receiver once he decrypts using the sender's key he will know that is sent form the sender. So in order to send data through a secure channel we use used SSL which utilizes public key cryptography to encrypt the data. So if SSL uses public key cryptography to encrypt the data, why is a certificate necessary?
\par The technical answer to that question is that a certificate is not really necessary, the data is secure and cannot easily be decrypted by a third party. However, certificates do serve a crucial role in the communication process. The certificate, signed by a trusted Certificate Authority (CA), ensures that the certificate holder is really who he claims to be. Without a trusted signed certificate, your data may be encrypted, however, the party you are communicating with may not be whom you think. Without certificates, impersonation attacks would be much more common. As theses certificates contains information about the owner of the certificate, like e-mail address, owner's name, certificate usage, duration of validity, resource location or Distinguished Name (DN) which includes the Common Name (CN) such as web site address or e-mail address, and the certificate ID of the person who certifies (signs) this information. It contains also the public key and finally a hash to ensure that the certificate has not been tampered with.